% Appendix Z converted from appendix_z_consolidated_figures.md
\chapter{Consolidated Figures, Diagrams, and Visual Content}

\textbf{Purpose and Justification}: This appendix consolidates all figures, diagrams, tables, and visual content referenced throughout the Multi-Sensor Recording System thesis. By centralising visual materials, this appendix enables efficient cross-referencing while maintaining narrative flow in the main chapters. The consolidation demonstrates the comprehensive nature of the visual documentation supporting the research while providing a single reference point for all graphical content.

% NOTE: Content preserved via verbatim for speed and fidelity. TODO: Convert key tables to LaTeX tabular and render Mermaid to static images for print in a later pass.

% Rendered Mermaid examples (compiled by LaTeX). See README for build requirements.


\section*{Rendered Mermaid Diagrams (LaTeX)}

\begin{mermaid}
    flowchart TD
    A[Researcher] -->|controls| PC[Desktop Controller]
    PC -->|start/stop| A1[Android Device 1]
    PC -->|start/stop| A2[Android Device 2]
    A1 -->|record| Store[(Local Storage)]
    A2 -->|record| Store
    PC -->|sync| NTP[NTP Clock]
\end{mermaid}

\begin{mermaid}
    sequenceDiagram
    participant R as Researcher
    participant PC as Controller
    participant A as Android
    R->>PC: Start Recording
    PC->>A: start(timestamp)
    A-->>PC: ack(started)
    R->>PC: Stop Recording
    PC->>A: stop()
    A-->>PC: files(session.zip)
\end{mermaid}

\begin{verbatim}
This appendix provides a unified repository of all visual content for the Multi-Sensor Recording System thesis, organised by category and chapter origin for easy navigation and reference.

\subsection{Z.1 Chapter 2 Figures: Background and Literature Review}

\subsubsection{Z.1.1 Physiological Computing Context}

Figure 2.1: Emotion/Stress Sensing Modality Landscape

```mermaid
graph TD
    subgraph "Behavioural Modalities"
        FACE[Facial Expression Analysis]
        VOICE[Voice Stress Analysis]
        GESTURE[Body Language & Gestures]
        EYE[Eye Tracking & Gaze]
    end

    subgraph "Physiological Modalities"
        GSR[Galvanic Skin Response<br/>Contact-based]
        THERMAL[Thermal Imaging<br/>Contactless]
        HRV[Heart Rate Variability<br/>Contact/Contactless]
        EEG[Electroencephalography<br/>Contact-based]
    end

    subgraph "Hybrid Approaches"
        MULTIMODAL[Multi-Modal Fusion<br/>RGB + Thermal + GSR]
        CONTEXTUAL[Contextual Integration<br/>Environment + Behaviour]
    end

    FACE --> MULTIMODAL
    THERMAL --> MULTIMODAL
    GSR --> MULTIMODAL
    HRV --> CONTEXTUAL
    VOICE --> CONTEXTUAL
```

\textit{Figure 2.1: Emotion/Stress Sensing Modality Landscape showing the range of behavioural and physiological modalities available for stress detection research. Demonstrates the positioning of the multi-sensor approach within the broader landscape of affective computing technologies \cite{ref4}.}

Figure 2.2: Contact vs Contactless Measurement Pipelines

```mermaid
graph LR
    subgraph "Contact-Based Pipeline"
        CONTACT_SENSOR[Physical Electrodes<br/>GSR Sensor]
        CONTACT_SIGNAL[Direct Electrical<br/>Signal Acquisition]
        CONTACT_PROCESS[Signal Processing<br/>Filtering & Amplification]
        CONTACT_OUTPUT[GSR Measurement<br/>High Accuracy]
    end

    subgraph "Contactless Pipeline"
        CONTACTLESS_SENSOR[Thermal Camera<br/>RGB Camera]
        CONTACTLESS_SIGNAL[Optical Signal<br/>Extraction]
        CONTACTLESS_PROCESS[Computer Vision<br/>Feature Extraction]
        CONTACTLESS_ML[Machine Learning<br/>GSR Prediction]
        CONTACTLESS_OUTPUT[Predicted GSR<br/>Moderate Accuracy]
    end

    CONTACT_SENSOR --> CONTACT_SIGNAL
    CONTACT_SIGNAL --> CONTACT_PROCESS
    CONTACT_PROCESS --> CONTACT_OUTPUT

    CONTACTLESS_SENSOR --> CONTACTLESS_SIGNAL
    CONTACTLESS_SIGNAL --> CONTACTLESS_PROCESS
    CONTACTLESS_PROCESS --> CONTACTLESS_ML
    CONTACTLESS_ML --> CONTACTLESS_OUTPUT

    CONTACT_OUTPUT -.->|Ground Truth<br/>Training Data| CONTACTLESS_ML
```

\textit{Figure 2.2: Contact vs Contactless Measurement Pipelines illustrating the trade-offs between accuracy and intrusiveness in physiological monitoring approaches. Shows how contactless methods require machine learning to achieve GSR estimation \cite{ref1}.}

\subsubsection{Z.1.2 Physiological Response Mechanisms}

Figure 2.3: Stress Response Pathways

```mermaid
graph TD
    STRESSOR[Acute Stressor<br/>Cognitive Load, Social Stress]

    subgraph "Neural Processing"
        AMYGDALA[Amygdala<br/>Threat Detection]
        HYPOTHALAMUS[Hypothalamus<br/>Integration Centre]
    end

    subgraph "SAM Axis (Rapid Response)"
        SYMPATHETIC[Sympathetic Nervous System]
        ADRENAL_MED[Adrenal Medulla]
        CATECHOLAMINES[Epinephrine/Norepinephrine<br/>2-5 seconds]
    end

    subgraph "HPA Axis (Sustained Response)"
        CRH[CRH Release]
        PITUITARY[Anterior Pituitary]
        ACTH[ACTH Release]
        ADRENAL_CORTEX[Adrenal Cortex]
        CORTISOL[Cortisol Release<br/>15-30 minutes]
    end

    subgraph "Physiological Outputs"
        GSR_RESPONSE[GSR Response<br/>1-3 seconds]
        THERMAL_RESPONSE[Thermal Response<br/>15-30 seconds]
        CARDIOVASCULAR[Heart Rate Changes<br/>5-10 seconds]
    end

    STRESSOR --> AMYGDALA
    AMYGDALA --> HYPOTHALAMUS
    HYPOTHALAMUS --> SYMPATHETIC
    HYPOTHALAMUS --> CRH

    SYMPATHETIC --> ADRENAL_MED
    ADRENAL_MED --> CATECHOLAMINES
    CATECHOLAMINES --> GSR_RESPONSE
    CATECHOLAMINES --> CARDIOVASCULAR

    CRH --> PITUITARY
    PITUITARY --> ACTH
    ACTH --> ADRENAL_CORTEX
    ADRENAL_CORTEX --> CORTISOL
    CORTISOL --> THERMAL_RESPONSE
```

*Figure 2.3: Stress Response Pathways showing the dual pathway model of
stress response through SAM (Sympathoadrenal Medullary) and HPA
(Hypothalamic-Pituitary-Adrenal) axes. Demonstrates the temporal
characteristics of different physiological responses measured by the
system \cite{ref1,ref4}.*

Figure 2.4: Sensor Technology Characteristics Matrix

```mermaid
graph LR
    subgraph "Accuracy vs Intrusiveness"
        HIGH_ACC_HIGH_INT[Contact EEG<br/>High Accuracy<br/>High Intrusiveness]
        HIGH_ACC_LOW_INT[Shimmer GSR<br/>High Accuracy<br/>Medium Intrusiveness]
        MED_ACC_LOW_INT[Thermal Camera<br/>Medium Accuracy<br/>Low Intrusiveness]
        LOW_ACC_NO_INT[RGB Camera<br/>Lower Accuracy<br/>No Intrusiveness]
    end

    HIGH_ACC_HIGH_INT -.->|Research Grade| TARGET[Target Zone<br/>Research Applications]
    HIGH_ACC_LOW_INT -.->|Reference Standard| TARGET
    MED_ACC_LOW_INT -.->|Contactless Goal| TARGET
    LOW_ACC_NO_INT -.->|Mass Deployment| CONSUMER[Consumer Applications]
```

*Figure 2.4: Sensor Technology Characteristics Matrix illustrating the
trade-offs between measurement accuracy and user intrusiveness. Shows the
positioning of different sensor modalities and the research target zone
cite{ref8,ref16,ref20}.*

\subsubsection{Z.1.4 Multi-Modal Integration Rationale}

Figure 2.5: Multi-Modal Data Fusion Concept

```mermaid
graph TD
    subgraph "Data Sources"
        RGB[RGB Video<br/>30 fps<br/>Facial Features]
        THERMAL[Thermal Video<br/>25 fps<br/>Temperature Maps]
        GSR_REF[GSR Reference<br/>128 Hz<br/>Ground Truth]
    end

    subgraph "Feature Extraction"
        RGB_FEATURES[RGB Features<br/>- Heart Rate (rPPG)<br/>- Facial Expression<br/>- Motion Patterns]
        THERMAL_FEATURES[Thermal Features<br/>- Nasal Temperature<br/>- Periorbital Warmth<br/>- Breathing Patterns]
        GSR_FEATURES[GSR Features<br/>- Response Amplitude<br/>- Recovery Time<br/>- Baseline Level]
    end

    subgraph "Temporal Alignment"
        SYNC[Synchronisation Engine<br/>±2.1ms Precision]
    end

    subgraph "Machine Learning"
        FUSION[Feature Fusion<br/>Weighted Combination]
        PREDICTION[GSR Prediction<br/>Contactless Estimation]
    end

    RGB --> RGB_FEATURES
    THERMAL --> THERMAL_FEATURES
    GSR_REF --> GSR_FEATURES

    RGB_FEATURES --> SYNC
    THERMAL_FEATURES --> SYNC
    GSR_FEATURES --> SYNC

    SYNC --> FUSION
    FUSION --> PREDICTION

    GSR_FEATURES -.->|Training<br/>Supervision| FUSION
```

*Figure 2.5: Multi-Modal Data Fusion Concept showing how RGB video,
thermal imaging, and reference GSR data are combined through temporal
synchronisation and feature fusion to enable contactless GSR prediction.
Demonstrates the system's core research hypothesis cite{ref5,ref6,ref22}.*

\subsection{Z.2 Chapter 3 Figures: Requirements and Architecture}

\subsubsection{Z.2.1 System Requirements Overview}

Figure 3.1: Functional Requirements Hierarchy

```mermaid
graph TD
    SYSTEM[Multi-Sensor Recording System]

    subgraph "Primary Functions"
        RECORDING[FR1: Multi-Modal Recording]
        SYNC[FR2: Temporal Synchronisation]
        MANAGEMENT[FR3: Session Management]
    end

    subgraph "Supporting Functions"
        DEVICE[FR4: Device Integration]
        STORAGE[FR5: Data Storage]
        INTERFACE[FR6: User Interface]
        CALIBRATION[FR7: Sensor Calibration]
    end

    subgraph "Quality Assurance"
        VALIDATION[FR8: Data Validation]
        RECOVERY[FR9: Error Recovery]
        EXPORT[FR10: Data Export]
    end

    SYSTEM --> RECORDING
    SYSTEM --> SYNC
    SYSTEM --> MANAGEMENT

    RECORDING --> DEVICE
    RECORDING --> STORAGE
    SYNC --> INTERFACE
    MANAGEMENT --> CALIBRATION

    STORAGE --> VALIDATION
    INTERFACE --> RECOVERY
    CALIBRATION --> EXPORT
```

*Figure 3.1: Functional Requirements Hierarchy showing the organisation
of system requirements into primary functions, supporting functions, and
quality assurance measures. Demonstrates the comprehensive scope of the
system requirements defined in Chapter 3.*

\subsubsection{Z.2.2 Non-Functional Requirements Matrix}

Table 3.1: Non-Functional Requirements Specification

| Requirement Category | Specification | Target Value | Validation Method |
|---|---|---|---|
| Performance | | | |
| Synchronisation Accuracy | Clock offset tolerance | ±5ms | NTP measurement |
| Data Throughput | Multi-device recording | 45MB/s sustained | Network monitoring |
| Response Time | UI responsiveness | <200ms | User interaction tests |
| Reliability | | | |
| System Uptime | Continuous operation | >99.5% | Extended testing |
| Data Integrity | Recording completeness | >99.9% | Checksum validation |
| Error Recovery | Automatic reconnection | <5s recovery time | Fault injection |
| Usability | | | |
| Setup Time | Session preparation | <15 minutes | User studies |
| Learning Curve | Operator training | <2 hours | Competency tests |
| Error Rate | User mistakes | <1% | Task analysis |
| Security | | | |
| Data Encryption | Network communication | TLS 1.2+ | Security audit |
| Access Control | Authentication required | Token-based | Penetration testing |
| Privacy Protection | Data anonymisation | GDPR compliance | Ethics review |

*Table 3.1: Non-Functional Requirements Specification providing
quantitative targets and validation methods for system quality
attributes. Establishes measurable criteria for system evaluation.*

\subsubsection{Z.2.3 Use Case Scenario Diagrams}

Figure 3.2: Primary Use Case: Multi-Modal Recording Session

```mermaid
sequenceDiagram
    participant R as Researcher
    participant PC as Desktop Controller
    participant A1 as Android Device 1
    participant A2 as Android Device 2
    participant S as Shimmer GSR Sensor

    R->>PC: Create New Session
    PC->>PC: Generate Session ID
    PC->>A1: Discover Device
    PC->>A2: Discover Device
    A1->>PC: Device Capabilities
    A2->>PC: Device Capabilities

    R->>PC: Start Recording
    PC->>A1: Start Command + Timestamp
    PC->>A2: Start Command + Timestamp
    PC->>S: Start Streaming

    par Concurrent Recording
        A1->>A1: Record RGB + Thermal
        A2->>A2: Record RGB + Thermal
        S->>PC: Stream GSR Data
    end

    R->>PC: Stop Recording
    PC->>A1: Stop Command
    PC->>A2: Stop Command
    PC->>S: Stop Streaming

    A1->>PC: Transfer Files
    A2->>PC: Transfer Files
    PC->>PC: Validate Data Integrity
    PC->>R: Session Complete
```

*Figure 3.2: Primary Use Case sequence diagram for a multi-modal
recording session showing the interaction between researcher, controller,
devices, and sensors. Demonstrates the coordinated workflow for
synchronised data collection.*

\subsection{Z.3 Chapter 4 Figures: Design and Implementation}

\subsubsection{Z.3.1 System Architecture Diagrams}

Figure 4.1: System Architecture Overview

![Figure 4.1: System Architecture Overview](../../diagrams/fig_4_01_system_architecture_overview.png)

*Figure 4.1: System Architecture Overview showing the distributed
client-server architecture with PC controller coordinating multiple
Android sensor nodes. Illustrates the star topology with central
synchronisation and control cite{ref17,ref19,ref21}.*

Figure 4.2: Android Thermal Integration Data Flow

![Figure 4.2: Thermal Integration Flow](../../diagrams/fig_4_02_thermal_integration_flow.png)

*Figure 4.2: Android Thermal Integration Data Flow showing the complete
pipeline from USB thermal camera through Android processing to CSV
storage. Details the frame capture callback system and data formatting
\cite{ref16,ref20}.*

Figure 4.3: Shimmer GSR Integration Architecture

![Figure 4.3: Shimmer GSR Integration](../../diagrams/fig_4_03_shimmer_gsr_integration.png)

*Figure 4.3: Shimmer GSR Integration Architecture illustrating BLE
communication protocol and real-time data streaming with calibration and
CSV logging. Shows the complete sensor-to-storage pipeline \cite{ref8,ref15}.*

\subsubsection{Z.3.2 User Interface Design}

Figure 4.4: Desktop Controller GUI Layout

![Figure 4.4: Desktop GUI Layout](../../diagrams/fig_4_04_desktop_gui_layout.png)

*Figure 4.4: Desktop Controller GUI Layout showing the comprehensive
dashboard with device management, live preview panels, session controls,
and synchronisation status indicators. Demonstrates the research-grade
interface design \cite{ref17}.*

\subsubsection{Z.3.3 Communication and Data Processing}

Figure 4.5: Network Protocol Sequence

![Figure 4.5: Protocol Sequence](../../diagrams/fig_4_05_protocol_sequence.png)

*Figure 4.5: Network Protocol Sequence diagram detailing the JSON-based
communication protocol over WebSocket connections. Shows device
registration, command distribution, and data synchronisation messaging
\cite{ref21}.*

Figure 4.6: Data Processing Pipeline

![Figure 4.6: Data Processing Pipeline](../../diagrams/fig_4_06_data_processing_pipeline.png)

*Figure 4.6: Data Processing Pipeline illustrating the complete data flow
from multi-modal capture through storage, transfer, and export. Shows
concurrent processing streams and temporal alignment mechanisms \cite{ref22,ref23}.*

\subsection{Z.4 Performance and Diagnostic Analysis}

\subsubsection{Z.4.1 System Reliability and Error Analysis}

Figure G.1: Device Discovery Success Patterns

```mermaid
graph LR
    subgraph "Network Conditions"
        OPTIMAL[5GHz WiFi<br/>Dedicated Channel<br/><3m Range]
        GOOD[5GHz WiFi<br/>Shared Channel<br/>3-10m Range]
        POOR[2.4GHz WiFi<br/>Congested<br/>>10m Range]
    end

    subgraph "Discovery Success Rates"
        OPTIMAL_RATE[78% First Attempt<br/>94% Within 3 Attempts]
        GOOD_RATE[65% First Attempt<br/>89% Within 3 Attempts]
        POOR_RATE[45% First Attempt<br/>78% Within 3 Attempts]
    end

    OPTIMAL --> OPTIMAL_RATE
    GOOD --> GOOD_RATE
    POOR --> POOR_RATE
```

*Figure G.1: Device Discovery Success Patterns showing the relationship
between network conditions and device discovery reliability.
Demonstrates the importance of optimal network configuration for system
deployment \cite{ref19,ref21}.*

Figure G.2: Error Classification and Frequency

```mermaid
pie title System Error Distribution (1,679 Test Cases)
    "UI Threading Issues" : 34
    "Network Timeouts" : 28
    "USB Device Issues" : 15
    "Memory Management" : 12
    "Sensor Calibration" : 8
    "Data Corruption" : 3
```

*Figure G.2: Error Classification and Frequency showing the distribution
of error types encountered during comprehensive testing. UI threading and
network issues account for 62% of all system errors, informing
development priorities.*

Figure G.3: Error Recovery Success Rates

| Error Category | Automatic Recovery | Manual Intervention | Data Loss Risk |
|---|---|---|---|
| Network Disconnection | 94% within 2s | 6% require restart | <0.01% |
| USB Device Failure | 78% within 5s | 22% require reconnection | 0.05% |
| Memory Issues | 89% via GC | 11% require restart | 0.02% |
| Sensor Calibration | 95% auto-recal | 5% manual recal | 0% |
| Threading Issues | 67% automatic | 33% UI restart | 0% |

*Figure G.3: Error Recovery Success Rates demonstrating the system's
fault tolerance capabilities across different error categories. Shows
high automatic recovery rates with minimal data loss risk.*

\subsubsection{Z.4.2 Synchronisation Performance}

Figure 3.7: Temporal Synchronisation Accuracy

```mermaid
graph LR
    subgraph "Synchronisation Results"
        TARGET[Target: ±5ms]
        ACHIEVED[Achieved: ±2.1ms<br/>95th percentile: ±4.2ms]
        DRIFT[Drift: <0.1ms/minute]
    end

    subgraph "Performance Factors"
        NETWORK[Network RTT: 12.3ms avg]
        NTP[NTP Accuracy: ±1ms]
        PROCESSING[Processing Delay: 0.8ms]
    end

    TARGET -.->|Requirement| ACHIEVED
    NETWORK --> ACHIEVED
    NTP --> ACHIEVED
    PROCESSING --> ACHIEVED
```

*Figure 3.7: Temporal Synchronisation Accuracy showing achieved
performance versus requirements. The system exceeds synchronisation
targets by 2.4× margin, ensuring research-grade temporal precision.*

Figure 3.8: Synchronisation Quality vs Network Latency

```mermaid
graph LR
    subgraph "Network Quality Impact"
        LOW_LAT[<20ms RTT<br/>±1.2ms sync error]
        MED_LAT[20-50ms RTT<br/>±2.1ms sync error]
        HIGH_LAT[50-100ms RTT<br/>±4.8ms sync error]
        POOR_LAT[>100ms RTT<br/>±12.3ms sync error]
    end

    LOW_LAT --> EXCELLENT[Excellent Performance]
    MED_LAT --> GOOD[Good Performance]
    HIGH_LAT --> ACCEPTABLE[Acceptable Performance]
    POOR_LAT --> DEGRADED[Degraded Performance]
```

*Figure 3.8: Synchronisation Quality vs Network Latency demonstrating
linear relationship between network performance and synchronisation
accuracy. Validates the <100ms network requirement specification.*

\subsubsection{Z.4.3 Data Quality and Sensor Performance}

Figure 3.9: GSR Sampling Rate Stability

```mermaid
graph LR
    subgraph "Sampling Performance"
        TARGET_RATE[Target: 128 Hz]
        ACTUAL_RATE[Actual: 127.98 Hz<br/>±0.12 Hz StdDev]
        DROPOUT[Sample Loss: <0.1%]
    end

    subgraph "Quality Metrics"
        SNR[SNR: 28.3 ± 3.1 dB]
        BASELINE[Baseline Stability: ±0.008 μS]
        SENSITIVITY[Event Detection: 94.7%]
    end

    TARGET_RATE --> ACTUAL_RATE
    ACTUAL_RATE --> DROPOUT
    DROPOUT --> SNR
    SNR --> BASELINE
    BASELINE --> SENSITIVITY
```

*Figure 3.9: GSR Sampling Rate Stability showing consistent performance
at the target 128 Hz with minimal sample loss and high signal quality.
Demonstrates research-grade data acquisition capabilities \cite{ref8}.*

\subsubsection{Z.4.4 Operational Metrics}

Figure 3.10: Session Workflow Timeline

```mermaid
gantt
    title Typical Recording Session Timeline
    dateFormat X
    axisFormat %M:%S

    section Setup Phase
    Power Management          :a1, 0, 2m
    Network Verification      :a2, after a1, 1m
    Device Registration       :a3, after a2, 2m
    Calibration              :a4, after a3, 3m

    section Recording Phase
    Session Preparation      :b1, after a4, 1m
    Active Recording         :b2, after b1, 30m
    Quality Monitoring       :b3, after b1, 30m

    section Completion Phase
    Stop Coordination        :c1, after b2, 1m
    Data Transfer           :c2, after c1, 3m
    Validation              :c3, after c2, 2m
```

*Figure 3.10: Session Workflow Timeline showing the typical duration and
sequencing of recording session phases. Total session overhead averages
15 minutes for 30-minute recording sessions.*

Figure 3.11: System Reliability Timeline

```mermaid
graph LR
    subgraph "72-Hour Continuous Operation"
        HOUR_0[Hour 0<br/>System Start<br/>100% Performance]
        HOUR_24[Hour 24<br/>Memory: 1.1GB<br/>CPU: 12.8%]
        HOUR_48[Hour 48<br/>Memory: 1.2GB<br/>CPU: 12.3%]
        HOUR_72[Hour 72<br/>Memory: 1.2GB<br/>CPU: 12.1%]
    end

    HOUR_0 --> HOUR_24
    HOUR_24 --> HOUR_48
    HOUR_48 --> HOUR_72

    subgraph "Performance Metrics"
        UPTIME[99.97% Uptime]
        NO_LEAKS[No Memory Leaks]
        STABLE_CPU[Stable CPU Usage]
    end

    HOUR_72 --> UPTIME
    HOUR_72 --> NO_LEAKS
    HOUR_72 --> STABLE_CPU
```

*Figure 3.11: System Reliability Timeline from 72-hour endurance testing
showing stable resource utilisation and sustained performance.
Demonstrates production-ready system stability.*

\subsubsection{Z.4.5 Network and Storage Performance}

Figure 3.12: Data Throughput Analysis

```mermaid
graph TD
    subgraph "Data Volume Distribution"
        RGB[RGB Video: 68%<br/>1.56 GB/30min]
        THERMAL[Thermal Data: 23%<br/>0.53 GB/30min]
        GSR[GSR CSV: 4%<br/>0.09 GB/30min]
        LOGS[Sync Logs: 3%<br/>0.08 GB/30min]
        META[Metadata: 2%<br/>0.04 GB/30min]
    end

    subgraph "Performance Metrics"
        WRITE_SPEED[Write Speed: 145 MB/s]
        TRANSFER_SUCCESS[Transfer Success: 99.2%]
        RETRY_RATE[Retry Rate: 3.1%]
    end

    RGB --> WRITE_SPEED
    THERMAL --> TRANSFER_SUCCESS
    GSR --> RETRY_RATE
```

*Figure 3.12: Data Throughput Analysis showing typical session data
breakdown and storage performance metrics. RGB video dominates storage
requirements while maintaining high transfer reliability \cite{ref23}.*

\subsection{Z.5 Cross-Reference Tables and Navigation Aids}

\subsubsection{Z.5.1 Figure Cross-Reference Table}

| Figure Number | Title | Original Location | Content Type | Key Topics |
|---|---|---|---|---|
| 2.1 | Emotion/Stress Sensing Landscape | Chapter 2.1 | Concept Diagram | Modality comparison |
| 2.2 | Contact vs Contactless Pipelines | Chapter 2.2 | Process Flow | Measurement approaches |
| 2.3 | Stress Response Pathways | Chapter 2.3 | Biological Model | Physiological mechanisms |
| 2.4 | Sensor Technology Matrix | Chapter 2.4 | Comparison Chart | Technology trade-offs |
| 2.5 | Multi-Modal Fusion Concept | Chapter 2.5 | System Model | Data integration |
| 3.1 | Functional Requirements | Chapter 3.2 | Hierarchy Diagram | System requirements |
| 3.2 | Recording Session Use Case | Chapter 3.5 | Sequence Diagram | Operational workflow |
| 4.1 | System Architecture | Chapter 4.1 | Technical Diagram | System structure |
| 4.2 | Thermal Integration | Chapter 4.2 | Data Flow | Implementation detail |
| 4.3 | GSR Integration | Chapter 4.2 | Architecture | Sensor integration |
| 4.4 | Desktop GUI | Chapter 4.3 | Interface Design | User experience |
| 4.5 | Protocol Sequence | Chapter 4.4 | Communication | Network protocol |
| 4.6 | Data Pipeline | Chapter 4.5 | Process Flow | Data management |
| G.1 | Device Discovery | Appendix G.1 | Performance Chart | Reliability analysis |
| G.2 | Error Classification | Appendix G.1 | Statistical Chart | Error analysis |
| G.3 | Error Recovery | Appendix G.1 | Performance Table | Fault tolerance |
| 3.7 | Sync Accuracy | Appendix G.2 | Performance Chart | Timing precision |
| 3.8 | Network vs Sync Quality | Appendix G.2 | Correlation Chart | Network impact |
| 3.9 | GSR Sampling | Appendix G.3 | Quality Metrics | Data quality |
| 3.10 | Workflow Timeline | Appendix G.4 | Process Chart | Operational efficiency |
| 3.11 | Reliability Timeline | Appendix G.4 | Endurance Chart | System stability |
| 3.12 | Throughput Analysis | Appendix G.5 | Performance Chart | Data handling |

\subsubsection{Z.5.2 Table Reference Index}

| Table Number | Title | Content | Location |
|---|---|---|---|
| 3.1 | Non-Functional Requirements | Quantitative specifications | Section Z.2.2 |
| G.3 | Error Recovery Rates | Statistical performance data | Section Z.4.1 |

\subsubsection{Z.5.3 Quick Navigation Guide}

By Chapter:
- Chapter 2 Figures: Sections Z.1.1 - Z.1.4 (Background concepts)
- Chapter 3 Figures: Sections Z.2.1 - Z.2.3 (Requirements and architecture)
- Chapter 4 Figures: Sections Z.3.1 - Z.3.3 (Implementation details)

By Content Type:
- Concept Diagrams: Figures 2.1, 2.3, 2.5, 3.1
- Process Flows: Figures 2.2, 4.6, 3.10
- Technical Architecture: Figures 4.1, 4.2, 4.3, 4.5
- Performance Analysis: Figures G.1-G.3, 3.7-3.12
- User Interface: Figure 4.4

By Research Domain:
- Physiological Computing: Figures 2.1, 2.3, 2.4
- System Architecture: Figures 3.1, 4.1, 4.5
- Implementation: Figures 4.2, 4.3, 4.6
- Validation: Figures G.1-G.3, 3.7-3.12

This consolidated visual content appendix provides comprehensive coverage
of all figures, diagrams, and tables referenced throughout the thesis,
enabling efficient navigation and supporting the modular documentation
approach adopted throughout the project.
\end{verbatim}
